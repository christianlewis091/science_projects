\section{Method}
\label{sec:method}

% symbols
% ^{\circ} percent
% ${\Delta^{14}C}$ D14C

The following institutions' radiocarbon measurements (${\Delta^{14}C}$ and/or FM) 
 were compared with the GNS Rafter Radiocarbon Laboratory in turn, each elaborated upon in the following sections. Each intercomparison is tailored specifically to the type of data available between each institution. 

\subsection{Rafter Radiocarbon Lab}

The rafter lab operates the longest runnnig record of atmospheric 14CO2...

\subsection{Heidelberg University}

The Heidelberg University Institute of Environmental Physics in affiliation with the ICOS Central Radiocarbon Laboratory operates a network of time-series stations measuring ${\Delta^{14}CO_{2}}$. One of these stations, Cape Grim, Tasmania (CGO; 40.68S, 144.68E, 94 m a.s.l; Levin et al., 2010), is a reasonable candidate through which to compare Heidelberg University to Rafter Radiocarbon Lab ${\Delta^{14}CO_{2}}$. Cape Grim and Baring Head observe a similar mixture of air from the Southern Ocean and Austrailia (Ziehn et al., 2014, Law et al, 2010), and a short initial time-series indicates no measurable differnece between the sites from 2017-2019 (~\ref{fig:bhdvcgo}). 

While the BHD record extends from 1950 to the present, the CGO record includes available data from 1987 to 2016, resulting in 30 years of overlapping data for intercomparison. 
Two intervals will are ignored: 1995-2006 and 2009-2012.
Variability in BHD exists between 1995 and 2005 as 1) the Rafter measurement method was changed from gas counting to AMS and 2) an online $^{13}C$ measurement allowed for approporiate fractionation correction (Turnbull et al, 2017; Zondervan et al., 2015), therefore this interval is ignored in the intercomparison. 
Additionally, the interval of 2009-2012 is ignored as the BHD record sees increasing noise in this period related to a temporary change in NaOH sampling protocol. Further details on the decision to remove these data can be found in the Supplementary Information. 
The remaining overlapping intervals are parsed into 4 sections: 1987 - 1991; 1991 - 1994; 2006 - 2009; 2012 - 2016. 
Each of these data intervals are non-stationary time-series containing seasonality. 
To extract long-term systematic offsets between institutions, and remove seasonality, the CCGCRV curve fitting procedure (Thoning et al., 1989; www.esrl.noaa.gov/gmd/ccgg/mbl/crvfit/) is implemented similar to (Turnbull et al., 2017). We employ the "smooth" and "trend" functions of the CCGCRV algoritm. "Smoothed" data includes the results of the polynomial and harmonic fits of the data, and a long-term low-pass filter of 667 days. "Trended" data is similar; retaining the polynomial fit to the function and ignoring harmonic components. 



%    def getSmoothValue(self, x):
%       """ Return the 'smoothed' data at time x
%     This is the function plus the smoothed residuals.
%   """

%    def getTrendValue(self, x):
%        """ Return the 'trend' of the data at time x
%        This is the polynomial part of the function plus the trend of the residuals.
%        i.e., poly plus the long term filter of the residuals
%        Values outside the range of x will be given a Nan
%        """


FFT filter cuttoff 1...
FTT Filter cutoff 2...

The data is broken up into 5 period in which the data overal: [1987 - 1991, 1991 - 1994, 2006 - 2016, 2006 - 2009, 2012 - 2016]....



























(including comparisons of standard material and non-stationary time series)  











In














\noindent where z is the water column depth (m), h is the mixed layer thickness (m), g=9.81 (m $\rm s^{-2}$) is the gravitational acceleration, $\bar{\rho}$ is the water column mean density determined from temperature and salinity profiles, $\rho(z)$ is the density profile determined from temperature and salinity 
profiles. 